\documentclass{article}
\usepackage{gonza}
\lfoot{\textcolor{gray2}{Copyright by G.A.}}

\lstset{language=c++,basicstyle=\footnotesize\ttfamily,
keywordstyle=\color{blue}\bfseries,frame=shadowbox}
\begin{document}
\title{Scientific Memos}
\author{G.A.}
\maketitle

\noindent

\section{Charge Gap}
\begin{equation}
C(N, N_e) \equiv E(N, N_e+1) + E(N, N_e-1) - 2 E(N, N_e),
\end{equation}
where $N_e$ is the number of electrons and $N$ the number of atoms.
Because the energy has to be extensive, we have
$E(N, N_e)=N\,e(N_e/N)$, where $e$ is an intensive function.
We take the limit of $N\rightarrow\infty$ with $N_e/N=n$ constant.
Therefore, \emph{as long as e'' exists}  \framebox{$C(N, N_e) \rightarrow e''(n)/N$.}

\section{Spin Gap}
\begin{equation}
S(N, N_\uparrow, N_\downarrow) \equiv (E(N, N_\uparrow+1, N_\downarrow-1) 
+E(N, N_\uparrow-1, N_\downarrow+1))/2 - E(N, N_\uparrow, N_\downarrow)
\end{equation}
where $N_\uparrow$ ($N_\downarrow$) is the number of electrons up (down) and $N$ the number of atoms.
Because the energy has to be extensive, we have
$E(N, N_\uparrow, N_\downarrow)=N\,e(n_\uparrow, n_\downarrow)$, where $e$ is an intensive function.
We take the limit of $N\rightarrow\infty$ with $N_\uparrow/N=n_\uparrow$ constant, and similar for $N_\downarrow$.
Therefore, \emph{as long as e'' exists}  \framebox{$S(N, N_e) \propto 1/N$.}


\section{The Binding Energy}
By definition, the binding energy
\begin{equation}
B(N, N_e) \equiv E(N, N_e+2) + E(N, N_e) - 2 E(N, N_e-1),
\end{equation}
where $N_e$ is the number of electrons and $N$ the number of atoms.
Therefore, \emph{as long as e'' exists}
\begin{equation}
E(N, N_e+2) = Ne(N_e/N + 2/N)=N\left[e(n_0)+\frac{2}{N}e'(n_0) + \left(\frac{2}{N}\right)^2e''(n_0)/2+\cdots\right],
\end{equation}
where $n_0\equiv N_e/N$. Finally \framebox{$B(N, N_e) = e''(n_0)/N$.}

\end{document}

